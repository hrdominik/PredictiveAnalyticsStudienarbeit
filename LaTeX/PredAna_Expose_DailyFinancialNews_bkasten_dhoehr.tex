% HIER BITTE AUSFÜLLEN:

\newcommand{\abschlussarbeit}{Exposé zur Studienarbeit} % Bachelorarbeit / Masterarbeit
\newcommand{\zweitpruefer}{Prof. Dr. Erika Mustermann} % Zweitprüfer

\newcommand{\sprache}{german} % Deutsch: german / Englisch: english
\newcommand{\druck}{oneside} %  / Simplex: oneside / Duplex: twoside

\newcommand{\titel}{Predictive Analytics \\
                    Daily Financial News for 6000+ Stocks}

\newcommand{\nameFirst}{Benjamin Kasten}
\newcommand{\nameSecond}{Dominik Höhr}
\newcommand{\matrikelnummerFirst}{7154784}
\newcommand{\matrikelnummerSecond}{7157684}
\newcommand{\adresseFirst}{Theodorstraße 22, 33102 Paderborn}
\newcommand{\adresseSecond}{Salentinstraße 16 A, 33102 Paderborn}
\newcommand{\upbmailFirst}{bkasten@mail.uni-paderborn.de}
\newcommand{\upbmailSecond}{dhoehr@mail.uni-paderborn.de}
\newcommand{\abgabedatum}{30. Mai 2020}


% % % % % % % % % % % % % % % % % % % % % % % % % % % % % % % % % % % % % % %


\documentclass[fontsize=12pt,BCOR=15mm,DIV=15,a4paper,headsepline,headings=small,\druck,openright,appendixprefix]{scrbook}

\usepackage[onehalfspacing]{setspace}
\raggedbottom
\usepackage{emptypage}
\makeatother 
\usepackage{parskip}

\usepackage[utf8]{inputenc} % Zeichenkodierung
\usepackage[T1]{fontenc} % Trennung von Wörtern mit Umlauten
\usepackage{lmodern} % Latin Modern font benutzen (enhanced computer modern clone, serif)
\usepackage[\sprache]{babel} % Dokument in deutsch inklusive Silbentrennung nach neuer deutscher Rechtschreibung
\usepackage{natbib} % Stil für Literaturverzeichnis
\usepackage{amsmath} % verbesserte mathematische Umgebung
\usepackage{amsfonts} % zusätzliche mathematische Zeichen
\usepackage{amssymb} % zusätzliche mathematische Zeichen
\usepackage{amsthm} % zusätzliche mathematische Zeichen
\usepackage[hyphens]{url}
\usepackage{exscale} % Anpassung mathematischer Symbole an die Schriftgröße
\usepackage{amstext} % \text in mathematischer Umgebung
\usepackage{array} % Tabellen
\usepackage{rotating} % rotierte Tabellen und Bilder
\usepackage[algoruled, algochapter]{algorithm2e} % Pseudocode
\usepackage{verbatim} % Funktion zum Schreiben von unformatiertem Text
\usepackage[section]{placeins} % Verhindert das Wandern von floats in eine andere section
\usepackage{color} % Farbiger Text
\usepackage{graphicx} % Funktionen zum Einfügen von Bildern
\usepackage{subfigure} % Anzeigen von Bildern aus mehreren Einzelbildern
\usepackage{setspace} % Einstellen des Zeilenabstandes
\usepackage[urlcolor = black,plainpages=false,pdfpagelabels=true,colorlinks=true,linkcolor=black,citecolor=black,bookmarksopen=true]{hyperref} % Verweise werden im PDF zu Hyperlinks
\usepackage{scrlayer-scrpage} % Koma für Kopfzeilen-Formatierung
\usepackage[nounderscore]{syntax}
\usepackage{tabulary} % bessere Tabellen
\usepackage{tabularx} % bessere Tabellen
\usepackage{longtable} % für Tabellen über mehrere Seiten
\usepackage{array}
\usepackage{paralist} % für kleine Einzüge mit \setdefaultleftmargin{0.5em}{}{}{}{}{}
\usepackage{booktabs}
\usepackage{multirow}
\usepackage[table,xcdraw]{xcolor}
\usepackage[hang]{footmisc} %Große Zahlen in Fußnoten
\usepackage{url}
\makeatletter
\def\@makefnmark{\hbox{\textsuperscript\@thefnmark}}
% \ohead{\leftmark} %Only chapter in header (not section!)
\makeatother
\let\footnotesize\small

\bibliographystyle{apalike} % Literaturverzeichnis entsprechend DIN 1505 formatiert, allerdings ohne URL und ISBN
\renewcommand{\topfraction}{0.85} % Anteil einer Seite, die von Floats belegt sein darf (default 0.7)
\renewcommand{\textfraction}{0.1} % Anteil einer Seite, die mindestens Text sein muss (default 0.2)
\renewcommand{\floatpagefraction}{0.75} % Anteil einer Seite, die ein Float einnehmen darf, ohne auf die nächste Seite gesetzt zu werden (default 0.5)

% Begin
\begin{document}

% Front Matter
\pagestyle{empty}
\pagenumbering{alph}

\vfil

\begin{titlepage}

        \begin{center}
        	\includegraphics[scale=1.]{img/data_analytics_group.png}
        \end{center}
        
        \vspace{8ex}
        
        \begin{center}
            
            Universität Paderborn\\
            Fakultät für Wirtschaftswissenschaften\\
            Department Wirtschaftsinformatik\\
            
            \vspace{8ex}
            
            \Large
            \abschlussarbeit\\
            
            \vspace{7ex}
            
            \textbf{\sffamily{\titel}}
            
            \vspace{7ex}
            
            \normalsize
            
            by\\
            \name\\
            Matrikelnummer:~\matrikelnummer\\
            \adresse\\
            \upbmail\\
            
            \vspace{8ex}
            
            vorgelegt bei\\
            Prof. Dr. Oliver Müller\\
            \zweitpruefer\\
            
            \vspace{10ex}
            
            \abgabedatum
            
        \end{center}
        
    \end{titlepage}
    
\vfil 
\cleardoublepage

\pagestyle{scrheadings}
\clearscrheadfoot
\ohead{\headmark}
\ofoot{\pagemark}

\frontmatter

% Expose
\mainmatter
\chapter*{Expose}
\section*{Forschungsfrage und Thema}

Zu Grunde gelegter Datensatz: 
Daily Financial News for 6000+ Stocks
% \\ see: \url{https://www.kaggle.com/miguelaenlle/massive-stock-news-analysis-db-for-nlpbacktests} \\ \\
\citep[see][]{dailyfinancialnews} \\
Der Datensatz liefert rund 1.410.000 Überschriften von Internetartikeln, über die Entwicklung von zugeordneten Aktien, welche an der Amerikanischen Börse notiert sind.\\
Zu beachten ist, dass die Internetartikel hauptsächlich von auf benzinga.com veröffentlicht wurden, dabei bestehen die Artikel häufig nur aus den Überschriften oder einer Sammlung von aktuellen Zahlen der Aktie(n).

Ziel der Studienarbeit ist es, die Headlines im ersten Schritt mittels eines Algorithmus (bspw. Topic Modeling oder Sentiment Analyse) zu klassifizieren oder zu kategorisieren, und anschließend zu untersuchen, ob das Ergebnis einem objektivem Bewertungskriterium folgt und beispielsweise mit dem Image, dem Aktienkurs oder den Umsatz assoziiert werden kann.
Zur Findung eines Bewertungskriterium müssen weitere Datensätze ergänzt werden.

\section*{Wie sind Sie auf das Thema gekommen? Welche Hintergrundgeschichte führte zu der Forschungsfrage?}
Der Markt wird immer schnelllebiger, insbesondere der Aktienmarkt erlebt einen immer schnelleren Wandel.\\
Die Einflussnahme von Image und Schlagzeilen nimmt dabei ebenfalls ständig zu und die Auswirkungen sind nicht immer vollständig zu erfassen.
Für Unternehmen ist es von essentieller Bedeutung zu wissen, wie sich Headlines langfristig auf sie selber auswirken. 

\section*{Warum ist dieses Thema bzw. diese Fragestellung untersuchenswert?}
Dieses Thema ist untersuchenswert, Unternehmen in der Öffentlichkeit stehen und abhängig von Kund:innen sind, die Nachrichten kosumieren. Dementsprechend beeinflussen Nachrichten das Kaufverhalten und damit das Unternehmen. Durch das steigende INteresse der Menschen an Anlagestrategien, werden auch Aktiennews immer wichtiger. Deshalb ist es entscheiden für Unternehmen, erkennen zu können, was für einen Effekt Headlines auf KPIs haben.

\section*{Wer sind die möglichen Akteure? Wer könnte von unserer Forschung profitieren?}
Die Akteure sind die börsennotierten Unternehmen. Diese können durch die Analyse der Headlines, die ihr Unternehmen betreffen, langfristige Trends erkennen, und so präventiv Maßnahmen treffen, um negative Auswirkungen auf ihre Unternehmen durch bestimmte Headlines zu verhindernn.


\section*{Was wurde in diesem oder einem ähnlichen Gebiet bereits getan?}
Es wurden bereits einige Analysen bezüglich der Auswirkungen von Headlines auf den Aktienkurs gemacht. Basierend auf Headlines aus der Vegangenheit und der darauf folgenden Kursenwticklung soll vorhergesagt werden, wie sich der Aktienkurs entwickelt, mit dem Ziel, Investitionsentscheidungen etc. zu erleichtern.\\

Wir möchten abschließend darauf hinweisen, dass eine konkrete Bewertung der Überschriften anhand der Aktienentwicklung nur mit entsprechenden Finanzdaten möglich ist. Die entsprechenden APIs sind dabei in der Größenordnung des Datensatzes nur kostenpflichtig zu erreichen.
% \\ see: \url{https://www.kaggle.com/danielstegemann/yahoofinance} \\

\citep[see][]{yahoofinance} \citep[see][]{econ_topicmodelling}


% Bibliography
\newpage
\addcontentsline{toc}{chapter}{Bibliography} 
\bibliography{bibliography}

% End
\end{document}