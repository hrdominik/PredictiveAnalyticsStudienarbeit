\chapter{Evaluation} % @bkasten

\section{Ergebnissevaluierung im Hinblick auf die Geschäftsziele}
Wie im Kapitel Business Understanding beschrieben, sind die Auswirkungen von Headlines sowohl für die betroffenen Unternehmen als auch (private) Aktienhändler interessant. Blicken wir zusätzlich auf die Forschungsfrage zurück, sind mehrere Ergebnisse zu evaluieren.\\
Grundsätzlich können wir mit unseren Ergebniss mit einer etwa 80-prozentigen Wahrscheinlichkeit anhand einer Headline bestimmen, ob sich der Aktienkurs positiv oder negativ entwickelt. Wir können jedoch nur mit deutlich geringerer Wahrscheinlichkeit die genaue Änderung des Aktienpreises vorhersagen. \\
Die Frage ist nun, ob es für die Geschäftsziele ausreicht, den Trend vorherzusagen. Für Anleger, die möglicherweise das Anlegen für andere übernehmen, wird die Analyse sehr wahrscheinlich nicht ausreichend sein. Eine begründete Entscheidung gegenüber dem Kunden für den Kauf einer Aktie kann möglicherweise nicht ausreichend aussagekräftig begründet werden. Für private Anleger, kann die schnelle Analysemöglichkeit jedoch zur Entscheidungsunterstütung herangezogen werden. Insbesondere unerfahrende Anleger können profitieren. Für Unternehmen, die die Auswirkung etwaiger Headlines einschätzen möchten, zeigt die Trendanalyse schnell und unkompliziert kritische Headlines, auf die mit unter anderem Marketing reagiert werden kann. Eine genaue Kalkulation des Aktienpreises ist jedoch nicht ohne weiteres möglich, konkrete Marktanalyse anhand der Headlines sind also nicht möglich. Außerdem zeigt der geringe Zusammenhang, des Sentiments, also der zugewiesenen Gefühle der Headline, zum Aktienpreis, dass letztendlich die Empfindung solcher Aktiennews subjektiv bleiben.\\
In Abgrenzung zur vorangegangen Literatur haben wir nur die Überschriften von Aktiennews betrachtet, mit einem offensichtlich schlechteren Ergebis als andere, die meist die vollständigen Artikel begutachteten. Daraus lässt sich ableiten, dass die Headline nicht ausreicht um eine fundierte Datenbasierte Entscheidung zutreffen, obwohl, wie zuvor literarisch gezeigt, Leser primär anhand der Headline agieren.
Schlussendlich können mit unseren Analysen grobe Empfehlungen gemacht werden, für die risikobehaftete Wirtschaft allerdings sind diese noch nicht ausreichend.

\section{Evaluierung des Prozesses}
Wir wenden in diesem Projekt das CRISP-DM Modell an. Durch die klare Differenzierung der einzelnen Schritte entsteht ein einheitliches und strukturiertes Vorgehen. Wir haben uns bei den einzelnen Schritten an der ausführlichen Beschreibung des CRISP-DM Modells aus \citep{crispdm} orientiert. Die einzelnen Schritte werden teils sehr ausführlich definiert, weshalb wir teilweise einige Abschnitte zusammengefasst haben oder ausließen. Im Gegensatz zu dem im CRISP-DM Modell vorgesehen Iterationen, haben wir nur sprunghaft Änderungen durchgeführt, die sich im Zusammenhang mit der Literaturrecherche und Empfehlungen einiger Kurse an der Universität ergeben haben.

\section{Ausblick}
Die beiden vorgestellten Ansätze zur Analyse von Headlines können noch deutlich intensiver einzeln betrachtet und weiter optimiert werden. Bei der Sentimentanalyse mittels eines Dictionary-Ansatzes, kann ein extra für den Finanzmarkt angepasstes Dictionary verwendet werden. Dieses müsste sich aber explizit auf die Aktiennews beziehen, und nicht auf Nachrichten aus der Allgemeinpresse. Für die darauf aufbauenden Predictions könnten unterschiedliche Algorithmen verwendet werden oder weitere Features mit einbezogen werden. Jedoch haben wir wie im Kapitel geschildert, feststellen müssen, dass ein wörterbuchbasierter Ansatz zur Vorhersage der Aktienpreisentwicklung eher ungeeignet ist. Der TF-IDF Ansatz mit Random Forest Regression schneidet deutlich besser ab. Hier lässt sich ein weitere Feintuning durchführen. Grundsätzlich könnten auch beide Ansätze kombiniert in einem Algorthimus betrachtet werden. Abschließend bleibt zu erwähnen, dass die Datenbasis der evaluierten Modelle stark eingeschränkt sind. Sowohl die Historie als auch die Genauigkeit der Uhrzeiten waren dem Modell, aufgrund äußerer Umstände, nicht gegeben. Rein spekulativ ist auch der verwendete Datensatz der Headlines nicht vollständig aussagekräftig gegenüber dem Aktienmarkt, da hier nur ein Publisher betrachtet wurde und die Formulierungen der künstlerischen Freiheit der Journalisten hierdurch nicht die Gesamtheit abbildet.