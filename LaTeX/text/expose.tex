\chapter{Expose}
\section{Forschungsfrage und Thema}

Zu Grunde gelegter Datensatz: 
Daily Financial News for 6000+ Stocks
% \\ see: \url{https://www.kaggle.com/miguelaenlle/massive-stock-news-analysis-db-for-nlpbacktests} \\ \\
\citep[see][]{dailyfinancialnews} \\
Der Datensatz liefert rund 1.410.000 Überschriften von Internetartikeln über die Entwicklung von zu geordneten Aktien, welche and er Amerikanischen Börse notiert sind.\\
Zu beachten ist, dass die Internetartikel hauptsächlich von auf benzinga.com veröffentlicht wurden, dabei bestehen die Artikel häufig nur aus den Überschriften oder einer Sammlung von aktuellen Aktien Entwicklung.

Angedacht ist eine Untersuchung der Zusammenhänge der Überschriften. \\
Dabei möchten wir auf die Wirkung und Auswirkung der Überschriften eingehen.

Die konkrete Forschungsfrage hängt von den ersten Ergebnissen ab.

\section{Hintergrund}
{\scriptsize \textit{Wie sind Sie auf das Thema gekommen? Welche Hintergrundgeschichte führte zu der Forschungsfrage?}}\\
Der Markt wird immer schnelllebiger, insbesondere der Aktienmarkt erlebt einen immer schnelleren Wandel.\\
Die Einflussnahme von Image und Schlagzeilen nimmt dabei ebenfalls ständig zu und die Auswirkungen sind nicht immer vollständig zu erfassen.
Dabei muss sowohl von der redaktionellen als auch der unternehmerischen Seite reaktiv schnell gehandelt werden.\\
Nach Möglichkeit der Forschungsergebnisse erhoffen wir uns zukünftig ein proaktives Handeln auf Aktienentwicklungen und Schlagzeilen Überschriften.

\section{Motivation}
{\scriptsize \textit{Warum ist dieses Thema bzw. diese Fragestellung untersuchenswert?}}\\
Die Entwicklung, Gruppierung und Formulierung der Überschriften sowie deren Mengenverhältnis zu den Aktien können interessante Erkenntnisse bringen.\\
Dabei wird die Analyse des redaktionellen Verhaltens in den Vordergrund gestellt.\\
Ist ein Zusammenhang zwischen den Überschriften erkennbar, so können weitere Forschungsfragen angeschlossen werden.\\ \\
Zum Beispiel kann im Anschluss die Entwicklung des Marktpreise der Aktien, aufgrund der kategorisierung der Überschrift,\\ 
oder die Entwicklung der Ausdruckweise der Headlines, dabei geht es primär um das Image des börsennotierten Unternehmens, untersucht werden.\\
Die Abhängigkeiten der Aktienkurse zu den Überschriften können dann sowohl proaktiv als auch reaktiv umgesetzt werden und in den Unternehmen oder in den Redaktionen zum Einsatz kommen.


\section{Akteure und Profit}
{\scriptsize \textit{Wer sind die möglichen Akteure? Wer könnte von unserer Forschung profitieren?}}\\
Die entsprechenden börsennotierten Unternehmen könnten ein berechtigtes Interesse an deren Kategorisierung der Überschrift mit anderen Wettbewerbern haben.
Dabei ist das Image des Unternehmens zu nennen, welches durch gemeinsame kategorisierung mit anderen Unternehmen auf dem Markt stark beeinflusst werden kann.
Wie oben schon einleitend genannt ist es auch denkbar die Marktpreise heranzuziehen, hier ist es im Interesse des börsennotierten Unternehmens einer positive Markt Entwicklung gegenüber zu stehen.

\section{Bisher}
{\scriptsize \textit{Was wurde in diesem oder einem ähnlichen Gebiet bereits getan?}}\\
Einige Anstrengungen, insbesondere mit dem zu Grunde gelegten Datensatz, ergänzten den Datensatz mit den Finanzdaten der genannten Aktien.
% \\ see: \url{https://www.kaggle.com/danielstegemann/yahoofinance} \\

\citep[see][]{yahoofinance}