\chapter{Business Understanding}
\section*{Motivation}
Seit der Digitalisierung und dem Zugang zum Internet für den Großteil der Weltbevölkerung werden Nachrichten fast schon in Echtzeit auf Onlineportalen veröffentlicht und frei zugänglich gemacht. So ist es jeder Person mit Internetzugang möglich, aktuelle Nachrichten fast ohne Zeitverzug parallel zu einer riesigen Menge anderen Menschen zu konsumieren, und sich selber mittels Kurznachrichten in sozialen medien selbstädnig überall schnell zu äußern, um seine Meinung oder eine Information öffentlich zu machen \citep{agarwel2016}. Darunter fallen auch Nachrichten bezüglich Aktien (im Folgenden auch Aktiennews). Aktiennews sind Nachrichten, die sich mit der Entwicklung von Aktienpreisen beschäftigen. Zu einer Aktiennews gibt es immer eine Headline. In dieser Headline versuchen die Autor:innen des Artikels, das Thema der Nachricht kurz und knapp zu beleuchten und zusammenzufassen. So beinhalten viele Headlines unter anderem die Änderung des Aktienpreises und das zugehörige Unternehmen. Durch die Headline wissen die Besucher eines jeweiligen Nachrichtenportals, wie sich ein Aktienkurs entwickelt, ohne den gesamten Artikel lesen zu müssen. Personen, die in Aktien, ETFs etc. investieren und Aktiennews lesen, reagieren möglicherweise direkt auf die aktuellste Headline, ohne den Artikel dazu zu lesen. Dabei haben Headlines häufig eine bestimmte Intention, die entweder postiv oder negativ zu einem bestimmten Thema steht. Solche Headlines können natürlich direkt die gesellschaft an sich beeinflussen \citep{agarwel2016} weshalb die Analyse von Aktiennews interessante Ergebnisse liefern kann. Beispielsweise kann auf eine negative Headline zu einem Unternehmen, dessen Aktien ein Kunde besitzt, zum schnellen Verkauf dieser führen. Positive Nachrichten könnten zu einem Anstieg der Nachfrage nach den Aktien des Unternehmens führen und somit dessen Wert erhöhen. Doch nicht immer muss die Headline die Entwicklung des Aktienkurses offenbaren, jedoch wird der Inhalt der Nachricht oft anhand der Headline bewertet. Eben dieser Umstand kann zu fehl Interpretationen führen \citep{dor2003}\\
Der Fokus auf die Headline an sich, ohne den restlichen Artikel, ist gerade deshalb so interessant, weil Menschen häufig einige Headlines überfliegen, statt den kopmletten Inhalt von Nachrichten zu lesen \citep{dor2003}. Headlines spielen also eine entscheidene Rolle bei der Meinungsbildung, sowohl auf die Sichtweise als auch die Emptionen, die in Zusammhang mit einem Produkt oder Szenario stehen. Selbst kürzeste Headlines spielen eine nicht zu vernachlässigende Rolle bei der Beurteilung des (möglichen) Inhalts einer Nachricht \citep{agarwel2016} \\ \\
Warum sollte jetzt ein Computer sich damit beschäftigen, Artikelheadlines zu analysieren, und zu bewerten, ob sich der Aktienkurs der in der Headline genannten Unternehmen verändern könnte? Natürlich könnten auch Menschen sich die Headlines durchlesen, und entsprechend darauf reagieren, bspw. mit einem Kauf oder Verkauf der Aktie. Jedoch können Computer um einiges schneller Texte verarbeiten, und somit auch schneller reagieren. Ein Computer kann also diese Arbeit für den Menschen übernehmen, und somit seinen Erfolg als Anleger erhöhen. Zudem ist es Möglich das ein entsprechender Algorithmus weitere Zusammenhänge erkennt, so wie zum Beispiel das Verhalten der Aktienkurse zu den veröffentlichten Überschriften. Es ist durch aus vorstellbar, das eben aufgrund der subjektive affektiven Handlung der Leser die Einschätzung zum Kaufen bzw. Verkaufen der Aktie aus historischer und kalkulierter Sicht falsch sind.

\section*{Forschungsfrage}
Wir wollen untersuchen, ob es einen Zusammenhang zwischen den vorkommenden Wörtern in den Headlines und dem Aktienkurs nach der Veröffentlichung der Headline gibt. Lässt sich also anhand der Wörter und dem Sentiment einer Headline die Veränderung des Aktienkurs, auf die sich die Headline bezieht, vorhersagen?\\ \\
Die Sentiment Analyse bezieht sich darauf, wie die Wörter eines Satzes (hier die Überschrift) aus Sicht des Menschen empfunden wird \citep[vgl.][]{agarwel2016}. Die Sentiment Analyse wird auf Basis von Sentiment-Dictionarys durchgeführt, diese beinhalten eine Zuordnung von Wörtern zu einem Score, welche die Gefühlslage des Wortes ausdrücken soll. Die Skala des Scores kann dabei ganz unterschiedlich aufgebaut sein, unter anderem: Positiv bis Negativ (5 bis -5) \citep{afinn} oder verschiedenen Gefühle entsprechend (anger, anticipation, disgust, fear, joy, sadness, surprise, trust) \citep{nrc}.
Innerhalb unserer Möglichkeiten prüfen wir, mit welchem Preis eine Aktie an dem Tag schließt, an dem die Headline veröffentlicht wurde. Grunflage dafür wäre ein Korrelation zwischen Sentiment Score und Stockpreisentwicklung.
\section*{Ziele und Profit}
Ziel ist es zukünftige Überschriften einordnen zu können. Damit die Auswirkungen einer veröffentlichten Headline einzuschätzen sind und gegebenenfalls Maßnahmen ergriffen werden können. \\
Dies hat sowohl für Unternehmen als auch für Inhaber der Aktien Vorteile. Die Unternehmen können so schnell den Überschriften entgegen wirken und den Aktienkurs somit stabil halten. Die Anleger haben aufgrund unseres Algorithmus eine gewisse Grundlage um nicht vorschnell, durch ihre Subjektive Auffassung der Überschrift, zu handeln.

\section*{Verwandte Literatur}
Das Thema Stock Price Prediction hat in der Literatur in den letzten Jahren einen großen Zuwachs bekommen. Hier listen wir einige Quellen auf, die in Zusammenhang mit unserem Forschungsthema stehen.\\
Lárló Nemes, Attila Kiss (2021) haben vier verschiedene Ansätze zur Sentiment Analyse von Economic News genutzt. Um die verschiedene Ansätze vergleichen zu können, wurde zu erst BERT genutzt, und dann die Tools Vader, Textblob und ein RNN. Es hat sich herausgestellt, dass BERT und RNN im Vergleich zum Vader Tool und Textblob deutlich einen deutlich besseren Senitment Score ermitteln konnten, ohne neutrale Ergebnisse. Darauf basierend konnte durch den Abgleich des Sentiments und der Stockpriceentwicklung der Moment festgestellt werden, der den Effekt der Headline auf den Aktienkurs abbildet.\\
Arul Agarwar (2020) hat mit Hilfe des Python Tools VADER eine Analyse der Nachrichten von einzelnen Unternehmen vorgenommen. Es wurden ganze HTML Seiten heruntergeladen,aus diesen die wichtigen Informationen herausgefiltert. Dann wurde ein wörterbuchbasierter Ansatz mithilfe des VADER Tools verwendet, um den einzelnen Nachrichten eine Sentiment Score zuzuweisen. Dem VADER LExicon wurden zudem weitere Wörter hinzugefügt bzw. das Sentiment geändert, um eine Missinterpretation (wegen dem Finanzmarkt) zu verhinden. Zwischen dem Sentiment der Nachrichten und dem Stock Price konnte eine starke Korrelation erkannt werden, die spätestens am nächsten Werktag auftrat.\\
Branko Kavsek (2017) stellte sich die Frage, ob es ein Wort oder eine Wortkombination gibt, deren Vorhandensein und deren Abwesenheit etwas über die Stockprice entwicklung aussagt. Dafür wurden unterschiedliche Modelle genutzt, die Teil des WEKA Tools sind. Bei allen Ansätzen wurde ein Overfitting festgestellt, da es einen starken Abfall der Accuracy beim Testset gab. Jedoch haben der PART, C4.5 und Random Forest Algorithmus sowohl eine 90\%tige Accuracy auf dem Trainingsset und jeweils mehr als 70\%tige Accuracy auf dem Testset. Das Ziel der Untersuchung war, bestimmte Wortkombinationen zu finden, im Anschluss wurden Entscheidungsregeln untersucht, die für die Bewertung der Stockpreise ausschlaggebend waren.\\
Ziniu Hu, Weiqing Liu, Jiang Bian, Xuanzhe Liu, and Tie-Yan Liu. 2018 (2018) haben für den Kontext der Vorhersage von Stockpreisen sogenannte Hybrid Attention Networks modelliert und implementiert. Grundlage hierfür sind die Schritte, die Menschen kognitiv durchführen, um mit chaotischen News umzugehen.\\
Kalyani Joshi, Prof. Bharathi H. N., Prof. Jyothi Rao (2016) haben unter der Annahme, das Stock News den Stockprice bestimmen, die Stockpriceentwicklung eines Unternehmen (Apple Inc.) anhand von Klassifikationsmodellen betrachtet. News (gesamte Artikel) wurden als positiv oder negativ eingestuft anhand ihrer Polarität. Genutzt wurde Random Forest, Naive Bayes und SVM. Mit einer Accuracy von 88\% bis 92\% abhängig von verschiedenen Test-Szenarien, wie unterschiedlicher Cross Validation, angepasstem Data Split oder ganz neuer Testdaten, konnte mittels Random Forest durch den Sentiment des Artikels der Stockpricetrend am besten vorhergesagt werden. \\
Saees Seifollahi und Mehdi Sharjari (2018) haben sich damit beschäftigt, Wörter zuerst in ihrem Kontext zu betrachten, bevor eine konventionelle Sentiment Analyse durchgeführt wird. Unter der Annahme, dass die Einordnung eines Wortes in den Kontext, der "Wortsinn", das Sentiment des jeweiligen Wortes überhaupt richtig identifiziert werden kann.

\section*{Nächste Schritte}
Im folgenden wollen wir zunächst einen Überblick über die vorhandenen Daten geben. Also den Inhalt sowie die Aussagekraft der Daten darlegen und erläutern. Außerdem auch abgrenzen was die Daten eben nicht aussagen. Dafür werden sowohl Metadaten des Datensatzes wie auch die eigentlichen Attribute des Datensatzes genau betrachtet. Hier werden noch keine Daten gefiltert, verändert oder ergänzt.\\
Vorab sei gesagt, dass der gegebene Datensatz keine Zielvariable enthält und diese somit noch nachträglich ergänzt werden muss. Dies führt zu einigen Einschränkungen. Diese Einschränkungen müssen entsprechend bewertet und eingeschätzt werden um die weitere Evaluation des zu trainierenden Modell zu ermöglichen.\\
Im Anschluss des Datenverständnis werden die Daten, im Zuge der Datenvorbereitung, so aufbereitet, dass diese vergleichbar und einheitlich sind. So dass das Modell die Daten entsprechend versteht. Außerdem wird der Datensatz so gefiltert, dass keine Lücken, Fehler oder nicht verwendbare Daten verbleiben. Dies geschieht vor allem aufgrund von Einschränkungen durch die Schnittstelle, welche benötigt wird um die fehlende Zielvariable zu ergänzen. Hier ist speziell darauf zu achten, das die Filterung des Datensatzes keinen, bzw. nur einen geringen, semantischen Einfluss aufweist.\\
Neben der Schnittstelle zu ''Polygon.io'', zum beziehen des Aktienkurses wird außerdem der Sentiment-Score der Headline im Datensatz ergänzt.\\
Nun gilt es, im Rahmen dieser Studienarbeit, ein Modell auf zubauen, welches unserer Forschungsfrage entspricht. Es wird also ein Modell entwickelt bzw. trainiert welches auf Basis des Sentiment-Scores einer Headline die Veränderung des Aktienkurses ins positive oder negative vorhersagen soll.\\
Zur Evaluation des Modells wird unter anderem die Accuracy betrachtet. Dies geschieht unter heranziehen unterschiedlicher Sentiment-Dictionarys.\\
Die gewonnen Erkenntnisse so wie das Modell werden im letzten Schritt hinsichtlich ihres Gewinn und Einsatzes für Wirtschaft und Wissenschaft analysiert.