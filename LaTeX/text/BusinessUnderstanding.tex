\chapter{Business Understanding}

% Hier soll das Problem erklärt werden, was bereits gemacht wurde und was von uns als nächstes getan wird
\section*{Motivation}
Seit der Digitalisierung und dem Zugang zum Internet für den Großteil der Weltbevölkerung werden Nachrichten fast schon in Echtzeit auf Onlineportalen veröffentlicht und frei zugänglich gemacht. So ist es jeder Person mit Internetzugang möglich, aktuelle Nachrichten fast ohne Zeitverzug parallel zu einer riesigen Menge anderen Menschen zu konsumieren. Darunter fallen auch Nachrichten bezüglich Aktien (im Folgenden Aktiennews). Aktiennews sind Nachrichten, die sich mit der Entwicklung von Aktienpreisen beschäftigen. Zu einer Aktiennews gibt es immer eine Headline. In dieser Headline versuchen die Autor:innen des Artikels, das Thema der Nachricht kurz und knapp zu beleuchten und zusammenzufassen. So beinhalten viele Headlines bspw. die Änderung des Aktienpreises und das zugehörige Unternehmen. Durch die Headline wissen die Besucher:innen eine Nachrichtenportals, wie sich ein Aktienkurs entwickelt, ohne den gesamten Artikel lesen zu müssen. Menschen, die in Aktien, ETFs etc. investieren und Aktiennews lesen, reagieren möglicherweise direkt auf die aktuelleste Headline, ohne den Artikel dazu zu lesen. Beispielsweise kann auf eine negative Headline zu einem Unternehmen, dessen Aktien ein Kunde besitzt, zum schnellen Verkauf dieser führen. Positive Nachrichten könnten zu einem Anstieg der Nachfrage nach den Aktien des Unternehmens führen und somit dessen Wert erhöhen.\\
Warum sollte jetzt ein Computer scih damit beschäftigen, Artikelheadlines zu analysieren, und zu bewerten, ob sich der Aktienkurs der in der Headline genannten Unternehmen verändern könnte? Natürlich könnten auch Menschen sich die Headlines durchlesen, und entsprechend darauf reagieren, bspw. mit einem Kauf oder Verkauf der Aktie. Jedoch können Computer um einiges schneller texte lesen, und somit auch schneller reagieren. Ein Computer kann also diese Arbeit für den Menschen übernehmen, und somit seinen Erfolg als Anleger:in erhöhen.
\section*{Forschungsfrage}
Wir wollen untersuchen, ob es einen Zusammenhang zwischen des Sentiment der Headlines gibt und dem Aktienkurs nach der Veröffentlichung der Headline. Falls ja, wann (zeitlich) lässt sich ein Effekt am stärksten feststellen? Lässt sich also anhand des Senitments einer Headline der Aktienkurs einer dieser Headline zu einem bestimten Zeitpunkt vorhersagen?
\section*{Verwandte Literatur}
Das Thema Stock Price Prediction hat in der LIterratur in den letzen Jahren einen großen Zuwachs bekommen. Hier listen wir einige Quellen auf, die in Zusammenhang mit unserem Forschungsthema stehen.\\
Lárló Nemes, Attila Kiss (2021) haben vier verschiedene Ansätze zur Sentiment Analyse von Economic News genutzt. Um die verschiedene Ansätze vergleichen zu können, wurde zu erst BERT genutzt, und dann die Tools Vader, Textblob und ein RNN. Es hat sich herausgestellt, dass BERT und RNN im Vergleich zum Vader Tool und Textblob deutlich einen deutlich besseren Senitment Score ermitteln konnten, ohne neutrale Ergebnisse. Daraufbasierend konnte durch den Abgelich des Senitments und der Stockpriceentwicklung der Moment festgestellt werden, der den Effekt der Haedline auf den Aktienkurs abbildet.\\
Arul Agarwar (2020) hat mit Hilfe des Python Tools VADER eine Analyse der Nachrichten von einzelnen Unternehmen vorgenommen. Es wurden ganze HTML Seiten heruntergeladen,aus diesen die bwichtigen Informationen herausgefiltert. Dann wurde ein wörterbuchbasierter Ansatz mithilfe des VADER Tools verwendet, um den einzelnen Nachrichten eine Sentiment Score zuzuweisen. Dem VADER LExicon wurden zudem weitere Wörter hinzugefügt bzw. das Sentiment geändert, um eine Missinterpretation (wegen dem Finanzmarkt) zu verhinden. Zwischen dem Sentiment der Nachrichten dem Stock Price konnte eine starke Korrelation erkannnt werden, die spätestens am nächsten Werktag auftrat.
\section*{Nächste Schritte}